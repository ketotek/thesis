\chapter{Design of the OpenFlow Bundles}
\label{chapter:architecture}  
% vim: set tw=78 sts=2 sw=2 ts=8 aw et ai:

One way of ensuring that multiple OpenFlow messages are correctly processed on several switches
is to execute them as part of a transaction. The idea is to create batches of messages and apply them
all in one step, in an atomic way. If any of the messages in a batch is found to be invalid, the entire
transaction is aborted.

This solution can become useful, but only if application controllers use it. To increase that chance, we
implemented it using a set of new OpenFlow 1.4 messages, Bundle messages.

The properties that our transaction system must have are atomic commits, consistency of changes and isolation.
The durability characteristic, commonly part of database transactions, is not of interest in this case, because
it is delegated to an actual database server, part of Open vSwitch.

Because of the high variety of OpenFlow messages that could be bundled together, we focus our design
and implementation only on some types of messages. These are messages that modify the state of the switch.
The Port modification messages are used to change various flags on the ports of a switch: link state,
dropping of incoming packets, link speed, auto-negotiation and many others. Flow entry modification messages
apply changes to the rules that a switch uses to make forwarding decisions. Other messages will have to
trigger an error if they are added to a bundle.

The high level algorithm that will govern the way bundles operate is shown in Listing~\ref{code:alg}. The top
level actions of opening a bundle, adding messages and committing will correspond to OpenFlow 1.4 bundle messages.
These will be sent by a controller application as part of the bundle workflow.

To ensure that at the time of the commit, the chances of failing are as small as possible, each message
is validated before being added to a bundle. These validity checks are specific to each message type and refer
both to the message format and content and to the switch state. The actual verification is done according to
the OpenFlow specification as if each message was received separately and not in a bundle.

\pagebreak
\begin{lstlisting}[caption={Basic Commit Algorithm}, label=code:alg]
Open a new bundle
Add a message
  Check message validity
  if invalid:
    drop bundle
  Push message to bundle
Close bundle
Commit bundle
  Create staging area
  Apply changes to staging area
  Queue notifications
  if no error:
     Change switch state to match staging area
     Send notifications
\end{lstlisting}

The isolation aspect of the transaction comes from the use of a staging area, separate from the current switch state.
When a message is decoded and added to a bundle, a copy of the data structure modified by the message is made.
All the changes will me made to this copy, without it affecting other parts of the switch. If multiple changes refer to
the same data structure, in this case, a port or a flow, the changes can be merged. Then, at the time of commit, only
the modifications between the state of the switch before the bundle and the final staging area state will be applied
in a single step.

One aspect that is very important both to the design and implementation of this bundles feature is the way we
treat side effects during the commit phase. Most of the side effects produced by applying a port or flow modification
message are alleviated by using the staging area. In case of errors during commit, we have $2$ options: we either do
nothing, if we didn't have side effects and just clear the staging area; or we rollback all the changes we made until
the error occurred. However, the second option is not enough. We could still have side effects, like sending status updates
which we can't rollback. The only solution is to delay these notifications until we are sure all the operations have
succeeded.



%\subsection{Implementation}

