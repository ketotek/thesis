\chapter{State of the Art}
\label{chapter:sota} 

The SDN Product Directory\cite{onf:products} page from Open Networking Foundation\cite{onf} lists
several products for hardware switching and virtual switching. Of all the virtual switching solutions,
only two of them mention supporting OpenFlow version 1.3. These are proprietary products that
we can't contribute to.

Another solution for virtual switching that is open source is Open vSwitch. It is able to connect to OpenFlow controllers,
but it does not support the newest additions to the protocol. The documentation mentions that 
``support for OpenFlow 1.1 and beyond is a work in progress''\footnote{\url{https://github.com/openvswitch/ovs/blob/master/OPENFLOW-1.1\%2B}}.
 
In order to work on new features for Open vSwitch, we must also analyze existing solutions for implementing controllers
that use OpenFlow 1.4.
 
\section{Related Work}

% TODO: Floodlight, OpenDaylight

On the controller side, there are also several implementations of frameworks used for developing
various network management and control applications. POX\cite{pox} can be used to write various networking software,
including OpenFlow switches and controllers. However, it can only communicate with OpenFlow 1.0 switches.

Another SDN framework is Ryu\cite{ryu}. It already supports various management protocols, including OpenFlow up to 1.4.
This can be a useful tool to validate our work on Open vSwitch. We will use it to create a simple learning switch
controller that can also send OpenFlow 1.4 messages.

\section{Project Motivation}

\section{Use cases}

Batching OpenFlow messages using bundles for having synchronized changes across multiple switches,
for example, during an initialization phase. The pre-validation of messages and atomic commit ensure
that the state remains consistent between the controller and several switches.

Another use case for batching messages is that of network security application controllers. These can range
from simple stateless firewalls to intrusion detection systems\cite{sdn}. They work by installing flows to
allow or drop future packets. The decision to do that can be static, based on access control list, or dynamic,
by inspecting the traffic patterns to determine the application types and permissions.
These types of security solutions can be built on top of OpenFlow controllers. By batching OpenFlow
messages we can ensure that security rules are enforced correctly and completely. In addition, we can
deploy a set of rules to multiple devices at the same time.