\chapter{Conclusion and Further Work}
\label{chapter:conclusion}  
 
% vim: set tw=78 sts=2 sw=2 ts=8 aw et ai:

Open vSwitch is an open source implementation of an OpenFlow software switch. While it is under active development,
support for OpenFlow is not complete. In this project we designed a couple of solutions that allow a better
synchronization between the state of the switch and the view the controller has about it. Firstly, we implemented
the possibility for a switch to notify a controller when it changes its permissions over the connection. Secondly,
we added the possibility to batch together multiple OpenFlow messages by implementing the bundles interface
from the OpenFlow specification.

The support for notifying controllers when a switch changes their role was done using OpenFlow \texttt{ROLE_STATUS}
asynchronous messages. Units tests were added to ensure that as few as possible regression will be possible.
These results were also sent for review to the open source development community and are now included in the
upstream version of Open vSwitch. There is work underway to update the asynchronous messages configuration to
OpenFlow 1.4. This will allow controllers to enable or disable the notifications for asynchronous messages
from switches, including \texttt{ROLE_STATUS} notifications.

There is now basic handling of OpenFlow 1.4 bundles messages in Open vSwitch. Our solutions allows a controller
to batch together port and flow modification messages and send them to execution to Open vSwitch. Port modifications
are merged together and applied as a single operation. Flow modification messages are only partially implemented
because of the complexity of the changes that they required. We only support adding and removing flows, and not
modifying existing flows. As far as contributing our work to the Open vSwitch project, the basic handling of bundle
messages has already been accepted. We are working on getting the actual commit for port-mod and flow-mod messages
in the right shape for upstream inclusion.

As the performance tests showed, our implementation of the bundles feature is, on average, $31\%$ slower than using
individual OpenFlow messages. This can be attributed to multiple factors, including the encapsulation of messages
and the need for multiple bundle control messages (for creating, closing and committing). Our solution does scale
well, in constant time to the number of ports and linearly to the number of messages.

Further work is still needed for our \texttt{ovs-pktgen} tool for better handling of various OpenFlow messages. At the
moment it is a side-project that resulted while developing OpenFlow bundles. Since many OpenFlow features are
not implemented in Open vSwitch, it may become a useful tool for others, so an upstream integration would be ideal.